% Inbuilt themes in beamer
\documentclass{beamer}

% Theme choice:
\usetheme{Boadilla}

\usepackage[framed]{seqcode}    % Also includes listings packae
%\usepackage{verbatim}

\usepackage{tikz}
\usetikzlibrary{calc}

% Title page details: 
\title{Sequence analysis tools}
%\subtitle{ISMRM virtual meeting: Vendor-Agnostic Pulse Sequence Programming with Pulseq: From Basics to Advanced Topics}
\author{Jon-Fredrik Nielsen}
\date{November 15, 2023}
\logo{\large \LaTeX{}}


\begin{document}

\beamertemplatenavigationsymbolsempty

% Title page frame
\begin{frame}
    \titlepage 
\end{frame}

% Remove logo from the next slides
\logo{}

\lstset{
    basicstyle=\tiny
}



% Outline frame
\begin{frame}{Outline}
    \tableofcontents
\end{frame}


%%%%%%%%%%%%%%%%%%%%%%%%%%%%%%%%%%%%%%%%%%%%%%%%%%%%%%%%%%%%%%%%%%%%%%%%%%%%%%%
\section{Example sequence: 2D spoiled GRE}
%%%%%%%%%%%%%%%%%%%%%%%%%%%%%%%%%%%%%%%%%%%%%%%%%%%%%%%%%%%%%%%%%%%%%%%%%%%%%%%
\begin{frame}[fragile]{Example sequence: 2D RF-spoiled gradient echo}

\begin{lstlisting}[language=MATLAB,frame=none]
    >> write2DGRE__;       % creates 'seq' object
    >> seq.plot('blockRange', [1 6], 'showBlocks', true);
\end{lstlisting}

\begin{center}
    \includegraphics<1>[width=.7\linewidth]{2DGRE.png}
    \includegraphics<2>[width=.7\linewidth]{2DGREzoom.png}
\end{center}

%\tikz[remember picture, overlay] \node[anchor=center] at ($(current page.center)+(0,-1.5)$) {\includegraphics[width=0.7\textwidth]{2DGRE.png}};

\end{frame}

\begin{frame}{Code excerpt}

\lstinputlisting[language=MATLAB]{write2DGRE__short.m}

\end{frame}



%%%%%%%%%%%%%%%%%%%%%%%%%%%%%%%%%%%%%%%%%%%%%%%%%%%%%%%%%%%%%%%%%%%%%%%%%%%%%%%
\section{seq.checkTiming()}
%%%%%%%%%%%%%%%%%%%%%%%%%%%%%%%%%%%%%%%%%%%%%%%%%%%%%%%%%%%%%%%%%%%%%%%%%%%%%%%
\begin{frame}[fragile]{seq.checkTiming()}

\begin{itemize}
    \item Checks timing of all blocks and objects in the sequence 
    \item Optionally returns the detailed error log as cell array of strings
    \item Usage:
    \lstinputlisting[language=MATLAB,frame=none]{checktiming__.m}
\end{itemize}
      
\bigskip

\end{frame}



%%%%%%%%%%%%%%%%%%%%%%%%%%%%%%%%%%%%%%%%%%%%%%%%%%%%%%%%%%%%%%%%%%%%%%%%%%%%%%%
\section{seq.testReport()}
%%%%%%%%%%%%%%%%%%%%%%%%%%%%%%%%%%%%%%%%%%%%%%%%%%%%%%%%%%%%%%%%%%%%%%%%%%%%%%%
\begin{frame}[fragile]{seq.testReport()}

\begin{itemize}
    \item Optional slow step, useful for testing during development
    \item Detects TE, TR, flip angle, gradient peak amp/slew, ...
    \item Usage:
        \begin{lstlisting}[language=MATLAB,frame=none]
        >> rep = seq.testReport;
        >> fprintf([rep{:}]);
        \end{lstlisting}
\end{itemize}
\end{frame}

\begin{frame}{seq.testReport()}
\lstinputlisting[language={},frame=none]{testreport.txt}
\end{frame}



%%%%%%%%%%%%%%%%%%%%%%%%%%%%%%%%%%%%%%%%%%%%%%%%%%%%%%%%%%%%%%%%%%%%%%%%%%%%%%%
\section{k-space analysis}
%%%%%%%%%%%%%%%%%%%%%%%%%%%%%%%%%%%%%%%%%%%%%%%%%%%%%%%%%%%%%%%%%%%%%%%%%%%%%%%
\begin{frame}{k-space analysis: seq.calculateKspacePP()}

Usage:
\vspace{-5mm}
\lstinputlisting[language=MATLAB,frame=none]{kspace__.m}

\begin{center}
    \includegraphics<1>[width=.7\linewidth]{kspace_full.png}
    \includegraphics<2>[width=.7\linewidth]{kspace.png}
\end{center}

\end{frame}



%%%%%%%%%%%%%%%%%%%%%%%%%%%%%%%%%%%%%%%%%%%%%%%%%%%%%%%%%%%%%%%%%%%%%%%%%%%%%%%
\section{Waveform shape analysis}
%%%%%%%%%%%%%%%%%%%%%%%%%%%%%%%%%%%%%%%%%%%%%%%%%%%%%%%%%%%%%%%%%%%%%%%%%%%%%%%
\begin{frame}{Waveform shape analysis}

\begin{itemize}
    \item Use seq.waveforms\_and\_times() to decompress waveforms
    \item Example usage:
        \vspace{-5mm}
        \lstinputlisting[language=MATLAB,frame=none]{waveformsandtimes__.m}
\end{itemize}

% Decompress the entire gradient waveform
            %   Returns gradient wave forms as a cell array with
            %   gradient_axes (typically 3) dimension; each cell contains
            %   time points and the correspndig gradient amplitude values.
            %   Additional return values are time points of excitations,
            %   refocusings and ADC sampling points.
            %   If the optionl parameter 'appendRF' is set to true the RF
            %   wave shapes are appended after the gradients
            %   Optional output parameters: tfp_excitation contains time
            %   moments, frequency and phase offsets of the excitation RF
            %   pulses (similar for tfp_refocusing); t_adc contains times
            %   of all ADC sample points; fp_adc contains frequency and
            %   phase offsets of each ADC object (not sample).
            %   TODO: return RF frequency offsets and RF waveforms and t_preparing (once its available)

\begin{center}
    \includegraphics<1>[width=.6\linewidth]{waveformsandtimes.png}
    \includegraphics<2>[width=.6\linewidth]{waveformsandtimes_zoom.png}
\end{center}


%seq.waveforms\_and\_times(), for example to create own plotting code, see for
%example writeEpiRS around line 188. Another interesting example would be to
%take a derivative of the gradient shapes to plot slew rates. A good learning
%is for instance to plot the vector norm of the gradient as a function of time
%for EPI, or even better the vector norm of the slew rate, showing that it is
%then easy to exceed the maximum slew rate limit by applying in-plane rotations
%to the EPI readout

\end{frame}

\begin{frame}[fragile]{Waveform shape analysis: combined gradient (norm)}

%\begin{lstlisting}[language=MATLAB,frame=none]
%    >> writeEpiRS__;     % creates 'seq' object
%\end{lstlisting}

\begin{center}
    \includegraphics<1>[width=.8\linewidth]{amp_norot.png}
    \includegraphics<2>[width=.8\linewidth]{amp_norot_norm.png}
\end{center}

\end{frame}

\begin{frame}[fragile]{Waveform shape analysis: combined slew rate (norm)}

\begin{center}
    \includegraphics<1>[width=.8\linewidth]{slew_norot.png}
    \includegraphics<2>[width=.8\linewidth]{slew_norot_norm.png}
\end{center}

\end{frame}


%\begin{frame}[fragile]{Oblique scan}
%TODO if time
%\end{frame}




%%%%%%%%%%%%%%%%%%%%%%%%%%%%%%%%%%%%%%%%%%%%%%%%%%%%%%%%%%%%%%%%%%%%%%%%%%%%%%%
\section{Slice profile simulation}
%%%%%%%%%%%%%%%%%%%%%%%%%%%%%%%%%%%%%%%%%%%%%%%%%%%%%%%%%%%%%%%%%%%%%%%%%%%%%%%
\begin{frame}[fragile]{Simulate slice profile}

\begin{itemize}
    \item Slice profile impacts SNR, partial volume effects, slice cross-talk
    \item Fortunately, easy to simulate. Example:
\end{itemize}

\lstinputlisting[language=MATLAB]{rfsim__.m}

\pause
\tikz[remember picture, overlay] \node[anchor=center] at ($(current page.center)+(0,0)$) {\includegraphics[width=0.8\textwidth]{slice.png}};

\end{frame}



%%%%%%%%%%%%%%%%%%%%%%%%%%%%%%%%%%%%%%%%%%%%%%%%%%%%%%%%%%%%%%%%%%%%%%%%%%%%%%%
\section{PNS}
%%%%%%%%%%%%%%%%%%%%%%%%%%%%%%%%%%%%%%%%%%%%%%%%%%%%%%%%%%%%%%%%%%%%%%%%%%%%%%%

\begin{frame}[fragile]{PNS modeling (GE)}

\begin{itemize}
    \item Nerve impulse response model
    \footnote{Schulte RF, Noeske R. Peripheral nerve stimulation‐optimal gradient waveform design. Magnetic resonance in medicine. 2015 Aug;74(2):518-22}
    \item Example:
\lstinputlisting[language=MATLAB,frame=none]{pnsdemo__.m}
\end{itemize}

\pause
\tikz[remember picture, overlay] \node[anchor=center] at ($(current page.center)+(0,0)$) {\includegraphics[width=0.8\textwidth]{pns.png}};

%\tikz[remember picture, overlay] \node[anchor=center] at ($(current page.center)+(0,-1.5)$) {\includegraphics[width=0.7\textwidth]{2DGRE.png}};
%\lstinputlisting[language=MATLAB]{write2DGRE__short.m}
%\begin{lstlisting}[language=MATLAB,frame=none]
%>> 47;
%\end{lstlisting}

\end{frame}


%%%%%%%%%%%%%%%%%%%%%%%%%%%%%%%%%%%%%%%%%%%%%%%%%%%%%%%%%%%%%%%%%%%%%%%%%%%%%%%
%\section{SAR: a simple reference measurement approach}
%%%%%%%%%%%%%%%%%%%%%%%%%%%%%%%%%%%%%%%%%%%%%%%%%%%%%%%%%%%%%%%%%%%%%%%%%%%%%%%
%\begin{frame}[fragile]{SAR: a simple reference measurement approach}

%TODO: experimental demonstration -- how accurate is it

%\end{frame}


\end{document}
